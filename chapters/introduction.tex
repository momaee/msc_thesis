% !TeX root=main.tex
% دستور زیر باید در اولین فصل شما باشد. آن را حذف نکنید!
\pagenumbering{arabic}

\chapter{مقدمه}
  \thispagestyle{empty}
    الگوی اینترنت اشیاء امکان اتصال اشیاء هوشمند به یکدیگر را ممکن می‌سازد.
    اتصال این دستگاه‌های هوشمند به یکدیگر باعث می‌شود که این دستگاه‌ها بتوانند با یکدیگر و با اینترنت داده مبادله کنند.
    این تبادل اطلاعات این امکان را ایجاد می‌کند که بتوانند خدمات متنوعی را برای کاربران فراهم کنند که قبلا امکان ارائه آن‌ها نبوده است.
    پیشرفت دستگاه‌های هوشمند و ظهور روش‌های جدید پردازش داده، اینترنت اشیاء را به عنوان گزینه‌ای مناسب برای استفاده در شهر هوشمند، شبکه هوشمند، خانه هوشمند و سلامتی هوشمند قرار داده است.
    شهر هوشمند از مهم‌ترین خدماتی است که توسط اینترنت اشیاء قابل تحقق است.
    به دلیل علاقه دولت‌‌ها برای استفاده از اینترنت اشیاء در بهبود مدیریت امور عمومی، شهر هوشمند به عنوان بهترین راهکار تحقق وسیع اینترنت اشیاء در نظر گرفته می‌شود.

    یکی از مهم‌ترین بخش‌های اینترنت اشیاء، پردازش داده‌هایی است که توسط حسگر‌های مختلف جمع‌آوری شده‌ اند.
    به طور سنتی پردازش ابری روشی کارا برای پردازش داده بوده است چرا که ظرفیت پردازشی ابر بسیار بیشتر از دیگر روش‌های پردازشی است.
    با این حال، کافی نبودن پهنای باند شبکه‌ها برای انتقال حجم انبوه داده‌های تولید شده در اینترنت اشیاء باعث می‌شود که انتقال داده‌ها به عنوان گلوگاه پردازش ابری باشد.
    به همین دلیل، ارسال همه‌ی داده‌ها برای پردازش به ابر، می‌تواند باعث زمان پاسخ طولانی سرویس‌ها بشود که برای بسیاری از سرویس‌ها قابل قبول نیست.
    از پردازش لبه به عنوان راه‌حلی برای این مشکل یاد می‌شود.
    در پردازش لبه هدف این است که پردازش داده‌ها تا حد ممکن در جایی که داده‌ها تولید می‌شوند انجام شود.

    در یک شبکه اینترنت اشیاء تعداد بسیار زیادی حسگر‌ها و فعال‌کننده‌ها مانند حسگر‌های دود، حسگر‌های دما، حسگر‌های حرکت دوربین‌های نظارتی و هشدار‌ها خطر آتش وجود دارند که در لبه شبکه قرار گرفته اند.
    همچنین سرویس‌های زیادی هم وجود دارند که از داده‌های این حسگر‌ها استفاده می‌کنند و با پردازش این داده‌ها، نتیجه‌هایی را تولید می‌کنند.
    برای هر سرویس، داده‌های حسگر‌ها باید به یک منبع پردازشی ارسال شوند و بعد از پردازش نتیجه به مقاصد مورد نظر فرستاده شوند.
    این مقاصد می‌توانند فعال‌کننده‌ها یا ذخیره‌سازهای ابری و غیره باشند.
    به عنوان نمونه می‌توان سرویس تشخیص آتش سوزی را نام برد.
    در این سرویس، ورودی‌ها می‌توانند حسگر‌های دود و دوربین‌های ویدیویی باشند و هشداردهنده‌‌های آتش خر وجی باشند.
    قسمت پردازش، ورودی حسگر‌ها و دوربین‌ها را مورد بررسی قرار می‌دهد و هشدار دهنده‌ها را فعال می‌کند یا می‌تواند به ایستگاه‌‌های آتشنشانی اطلاع دهد.
    به عنوان نمونه‌ی دیگر سرویس‌های امنیت ساختمان‌ها را در نظر بگیرید.
    در این سرویس‌ها، داده‌های حسگر‌های حرکتی و دوربین‌های نظارتی پردازش می‌شوند و در صورت تشخیص نفوذ غیر مجاز هشدار دهنده‌ها فعال می‌شوند و به پلیس اطلاع داده می‌شود.

    سرویس‌ها برای پردازش داده‌های خود باید منابع پردازشی مناسب را انتخاب کنند.
    تعداد بسیار زیاد سرویس‌ها و منابع پردازشی در شبکه اینترنت اشیاء باعث می‌شود که مسئله پیدا کردن منبع پردازشی بهینه برای سرویس‌ها، یک مسئله پیچیده باشد.
    به همین دلیل در این پایان نامه به بررسی و ارائه راه حل برای حل مسئله اختصاص منابع پرداشی در اینترنت اشیاء می‌پردازیم.

  \section{نوع‌آوری‌های پایان نامه}
    دستاورد‌های این پایان‌نامه را می‌توان به صورت زیر خلاصه کرد:

    در بخش اول، برای صورت مسئله تخصیص منابع به صورت یک به یک،مدل ریاضی ارائه شده‌است و مسئله به صورت یک بهینه سازی نوشته شده‌است که هدف آن کمینه کردن مجموع هزینه سرویس‌ها است. برای حل این مسئله یک روش تخصیص منابع مبتنی بر مزایده استفاده است که امکان حل این مسئله به صورت موازی را دارد که برای شبکه‌های بزرگ مناسب است.

    در بخش دوم، برای صورت مسئله تخصیص منابع به صورت چند به چند، مدل ریاضی ارائه شده و به صورت یک مسئله بهینه سازی فرمول بندی شده که نرخ ارسال داده‌ی حسگر‌ها و بیشترین تاخیر سرویس را در تابع هزینه سرویس‌ها دخیل می‌کند و مجموع هزینه سرویس‌ها را کمینه می‌کند.
    ارائه الگوریتم برای حل این مسئله و بررسی تاثیر نرخ انتخابی در تاخیر سرویس‌ها از دیگر نوع‌آوری‌های این پایان نامه است.

  \section{ساختار پایان‌نامه}
    ساختار پایان‌نامه به شرح زیر می‌باشد.

    \cref{chap:literature_review} به مروری بر مطالعات انجام شده در زمینه تخصیص منابع اینترنت اشیاء می‌پردازد.
    در این فصل، ابتدا به معرفی اینترنت اشیاء و شهر هوشمند می‌پردازیم.
    پس از آن برخی از خدماتی را که شهر هوشمند به کمک اینترنت اشیاء می‌تواند ارائه دهد، بررسی می‌کنیم.
    سپس به معرفی پردازش لبه می‌پردازیم و مزیت‌های آن را بررسی می‌کنیم.
    پس از آن به معرفی روش‌های مجازی سازی می‌پردازیم و مقالاتی که استفاده از مجازی سازی در لبه شبکه را مطالعه کرده‌اند، مرور می‌کنیم.
    در انتها، به بررسی مقالات منتشر شده در زمینه تخصیص منابع در اینترنت اشیاء می‌پردازیم.

    \cref{chap:one_to_one_allocation} به بررسی مسئله اختصاص منابع پردازشی به صورت یک به یک در شبکه اینترنت اشیاء اختصاص دارد.
    در این فصل فرض بر این است که سرویس‌ها، می‌توانند از یک منبع پردازشی استفاده کنند و منابع پردازشی هم می‌توانند به یک سرویس اختصاص پیدا کنند.
    تخصیص منابع به صورت یک مسئله بهینه سازی فرمول بندی می‌شود که حل بهینه‌ی آن پیچیده است.
    برای حل مسئله از یک روش مبتنی بر مزایده استفاده شده که جواب قابل قبولی برای مسئله ارائه می‌دهد.

    در \cref{chap:many_to_many_allocation} مسئله تخصیص منابع پردازشی به صورت چند به چند در شبکه اینترنت اشیاء مورد بررسی قرار گرفته است.
    در این فصل بر خلاف فصل قبل منابع پردازشی می‌توانند به چند سرویس اختصاص پیدا کنند و سرویس‌ها هم می‌توانند از چند منبع پردازشی استفاده کنند.
    مانند فصل قبل مسئله تخصیص منابع به صورت یک مسئله بهینه سازی فرمول‌بندی شده است که حل بهینه پیچیده‌ای دارد و برای حل آن الگوریتمی پیشنهاد شده که در زمان کوتاه‌تری مسئله را حل می‌کند.

    در انتها در \cref{chap:conclusion} به بیان نتیجه‌گیری و کار‌های آینده می‌پردازیم.
