% !TeX root=main.tex
% دستور زیر باید در اولین فصل شما باشد. آن را حذف نکنید!
\pagenumbering{arabic}

\chapter{مقدمه}
  \thispagestyle{empty}
	%todo    new subjects: all types of queuing systems and stability and queuing delay
	%todo sourced priorities
    اینترنت اشیاء(IoT\LTRfootnote{Internet of Things})امکان اتصال اشیاء هوشمند به یکدیگر را ممکن می‌سازد.
    اتصال این دستگاه‌های هوشمند به یکدیگر باعث می‌شود که این دستگاه‌ها بتوانند با یکدیگر و با اینترنت داده مبادله کنند.
    تبادل اطلاعات این امکان را ایجاد می‌کند که بتوانند خدمات متنوعی را برای کاربران فراهم کنند که قبلا امکان ارائه آن‌ها نبوده است.
    پیشرفت دستگاه‌های هوشمند و ظهور روش‌های جدید پردازش داده، اینترنت اشیاء را به عنوان گزینه‌ای مناسب برای استفاده در شهر هوشمند\LTRfootnote{Smart city} ، شبکه هوشمند انرژی\LTRfootnote{Smart grid}، خانه هوشمند\LTRfootnote{Smart home} و سلامتی هوشمند\LTRfootnote{Smart health} قرار داده است.
    امروزه یکی از مسائل پرطرفدار در اینترنت اشیاء، شهر هوشمند است که در آن هدف این است که کلیه وسائل موجود به نحوی هوشمند شده و ارتباطات بین آن‌ها به دور از نقش انسان و به صورت کاملا مکانیزه انجام شود.
    به دلیل علاقه دولت‌‌ها برای استفاده از اینترنت اشیاء در بهبود مدیریت امور عمومی، شهر هوشمند به عنوان بهترین راهکار تحقق وسیع اینترنت اشیاء در نظر گرفته می‌شود.

    یکی از مهم‌ترین بخش‌های اینترنت اشیاء، پردازش داده‌هایی است که توسط حسگر‌های مختلف جمع‌آوری شده‌ اند.
    به طور سنتی پردازش ابری\LTRfootnote{Cloud Computing} روشی کارا برای پردازش داده بوده است چرا که ظرفیت پردازشی ابر بسیار بیشتر از دیگر روش‌های پردازشی است.
    با این حال، افزایش چشمگیر در تعداد دستگاه‌های هوشمند و افزایش وسیع در میزان حجم اطلاعات موجود جهت پردازش و همچنین کافی نبودن پهنای باند شبکه‌ها برای انتقال حجم انبوه داده‌های تولید شده در اینترنت اشیاء باعث می‌شود که انتقال داده‌ها و پردازش همه‌ی آن‌ها در فضای ابری به عنوان گلوگاه پردازش ابری باشد.
    به همین دلیل، ارسال همه‌ی داده‌ها برای پردازش به فضای ابری، می‌تواند باعث زمان پاسخ طولانی سرویس‌ها شود که برای بسیاری از سرویس‌ها قابل قبول نیست.
    از پردازش لبه\LTRfootnote{Edge Computing} و پردازش مه\LTRfootnote{Fog Computing} به عنوان راه‌حل‌هایی برای این مشکل یاد می‌شود.
    در پردازش لبه هدف این است که پردازش داده‌ها تا حد ممکن در جایی که داده‌ها تولید می‌شوند انجام شود.
    پردازش مه نیز مشابهت زیادی به پردازش لبه دارد با این تفاوت که فاصله گره‌های پردازشی تا حسگرها نسبت به پردازش لبه بیشتر است. 

    در یک شبکه اینترنت اشیاء تعداد بسیار زیادی حسگر\LTRfootnote{Sensor}‌ و فعال‌کننده\LTRfootnote{Actuator}‌ مانند حسگر‌های دود\LTRfootnote{Smoke sensors}، حسگر‌های دما\LTRfootnote{Temperature sensors}، حسگر‌های حرکت\LTRfootnote{Motion sensors}، دوربین‌های نظارتی و هشدار‌های خطر آتش وجود دارند.
    همچنین سرویس‌های\LTRfootnote{Services} زیادی هم وجود دارند که از داده‌های این حسگر‌ها استفاده و با پردازش این داده‌ها، نتیجه‌هایی را تولید می‌کنند.
    برای هر سرویس، داده‌های حسگر‌ها باید به یک منبع پردازشی ارسال شوند و بعد از پردازش نتیجه به مقصد‌های مورد نظر فرستاده شوند.
    این مقصد‌ها می‌توانند فعال‌کننده‌ها یا ذخیره‌سازهای ابری و غیره باشند.
    به عنوان نمونه می‌توان سرویس تشخیص آتش سوزی را نام برد.
    در این سرویس، ورودی‌ها می‌توانند حسگر‌های دود،حسگرهای دما و دوربین‌های ویدیویی باشند و هشداردهنده‌‌های آتش، خروجی باشند.
    قسمت پردازش، ورودی حسگر‌ها و دوربین‌ها را مورد بررسی قرار می‌دهد و هشدار دهنده‌ها را فعال می‌کند یا می‌تواند به ایستگاه‌‌های آتشنشانی اطلاع دهد.  
    به عنوان نمونه‌ی دیگر، سرویس‌های امنیت ساختمان‌ها را در نظر بگیرید.
    در این سرویس‌ها، داده‌های حسگر‌های حرکتی و دوربین‌های نظارتی پردازش می‌شوند و در صورت تشخیص نفوذ غیر مجاز هشدار دهنده‌ها فعال می‌شوند و به پلیس اطلاع داده می‌شود.
    اطلاعات ورودی هر سرویس را می‌توان در قالب مجموعه‌ای از بسته‌ها و تحت عنوان وظیفه\LTRfootnote{Task} درنظر گرفت که هر وظیفه ویژگی‌ها و قیدهای مشخص و مخصوص به خود را دارد.
    
    هرچقدر تعداد اطلاعات ارسالی مربوط به یک محیط بیشتر باشد می‌توان گفت که احتمال بروز خطا در محاسبات کمتر می‌شود زیرا تصمیم‌گیری صرفا براساس اطلاعات مربوط یه یک حسگر انجام نمی‌شود.
    
    سرویس‌ها برای پردازش داده‌های خود باید در منابع پردازشی مناسب پردازش شوند به‌طوری‌که تمام قیدهای مربوط به آن‌ها به بهترین وجه ممکن برآورده شود. 
    تعداد بسیار زیاد سرویس‌ها و منابع پردازشی در شبکه اینترنت اشیاء باعث می‌شود که مسئله پیدا کردن منبع پردازشی بهینه برای سرویس‌ها، یک مسئله پیچیده باشد.
    به همین دلیل در این پایان نامه به بررسی و ارائه راه حل برای حل مسئله اختصاص منابع پردازشی در اینترنت اشیاء می‌پردازیم.

    \section{اینترنت اشیاء و شهر هوشمند}
    واژه‌ی اینترنت اشیاء برای اولین بار توسط کوین اشتون\LTRfootnote{Kevin Ashton} در یک ارائه برای استفاده از بازشناسی با امواج رادیویی\LTRfootnote{RFID} در مدیریت زنجیر تأمین\LTRfootnote{Supply Chain} استفاده شد\cite{shton2009that}.
    اینترنت اشیاء امکان اتصال هر کسی در هر مکان و زمانی به هر چیزی در هر مکانی و هر زمانی را فراهم می‌کند.
    با پیشرفت تکنولوژی به سمت جامعه‌ای پیش می‌رویم که همه افراد و همه‌ی اشیاء متصل خواهند بود\cite{zheng2011internet}.
    ایده‌ی اصلی اینترنت اشیاء این است که امکان اتصال خودکار و امن و انتقال داده‌ بین دستگاه‌های فیزیکی و برنامه‌های کاربردی را فراهم می‌کند.
    در واقع اینترنت اشیاء این امکان را ایجاد می‌کند که اشیاء فیزیکی بتوانند ببینند، بشنوند، و با صحبت کردن با یکدیگر بتوانند تصمیم‌گیری کنند و کار‌هایی را انجام دهند\cite{al2015internet}.
    در طول زمان انتظار می‌رود اینترنت اشیاء کاربرد‌های خانگی و تجاری فراوانی داشته باشد، کیفیت زندگی افراد را بهبود ببخشد و باعث رشد اقتصاد جهانی بشود.

    هدف اینترنت اشیاء این است که اینترنت را فراگیرتر و همه جانبه‌تر کند.
    علاوه بر این، به وسیله‌ی دسترسی آسان و تعامل با طیف گسترده‌ای از دستگاه‌هایی مانند لوازم خانگی، دوربین‌های نظارتی، حسگر‌ها، فعال‌کننده‌ها، نمایشگر‌ها، خودرو‌ها و غیره، اینترنت اشیاء به توسعه‌ی کاربرد داده‌های تولید شده توسط این دستگاه‌ها برای فراهم کردن خدمات به شهروندان، شرکت‌ها و اداره‌ی امور عمومی کمک می‌کند.
    این الگو، کاربرد‌هایی در حوزه‌های مختلف مانند اتوماسیون خانگی، اتوماسیون صنعتی، کمک‌های پزشکی، سلامت همراه، نگهداری از افراد سالمند، مدیریت هوشمند انرژی و شبکه هوشمند، وسایل نقلیه، مدیریت ترافیک و موارد دیگر دارد \cite{bellavista2013convergence}.
    
    در چنین عرصه‌ی ناهمگونی از کاربرد‌ها،پیداکردن یک راه‌حل که بتواند به همه‌ی نیاز‌های کاربرد‌های مختلف پاسخ بدهد، چالش برانگیز خواهد بود.
    دشواری پیدا کردن این راه‌حل باعث ایجاد راه‌حل‌های متفاوت و بعضاً ناسازگار برای تحقق سامانه‌های اینترنت اشیاء شده است \cite{zanella2014internet}.
    بنابر این از دید سیستمی، به دلیل نوآوری‌ها و پیچیدگی‌های اینترنت اشیاء، نیاز به تحقق یک شبکه‌ی اینترنت اشیاء، به همراه شبکه سرویس‌ها و شبکه دستگاه‌ها وجود دارد.
    علاوه بر مشکلات فنی، عدم وجود یک مدل تجاری مورد قبول که بتواند سرمایه‌گذاران را برای ترویج استقرار این تکنولوژی‌ها جذب کند مانع توسعه سریع استفاده از اینترنت اشیاء شده است\cite{laya2013investing}.

    در این سناریوی پیچیده، کاربرد اینترنت اشیاء در محیط‌های شهری، از محبوبیت بالایی برخوردار است چرا که به تقاضای بسیاری از دولت‌ها برای استفاده از فناوری‌ اطلاعات و ارتباطات در مدیریت امور عمومی پاسخ می‌دهد و باعث می‌شود مفهومی که از آن با عنوان شهر هوشمند یاد می‌شود، تحقق یابد \cite{schaffers2011smart}.
    در حالی که هنوز تعریفی که موردقبول همگان برای شهر هوشمند باشد وجود ندارد، هدف نهایی استفاده‌ی بهتر از منابع عمومی، افزایش کیفیت خدمات ارائه شده به شهروندان و کاهش هزینه‌های عملیاتی مدیریت امور عمومی است. 
    این اهداف به کمک اینترنت اشیاء شهری قابل دستیابی خواهند بود.
    منظور از اینترنت اشیاء شهری، زیرساخت ارتباطی است که دسترسی یکپارچه، ساده و اقتصادی را به مجموعه‌ای از خدمات عمومی فراهم می‌کند.
    اینترنت اشیاء شهری می‌تواند مزایایی در مدیریت و بهینه‌سازی خدمات عمومی مرسوم مانند حمل و نقل و پارک خودرو‌ها، روشنایی، نظارت و نگهداری اماکن عمومی، حفظ آثار باستانی، جمع آوری پسماند، بیمارستان‌ها و مدارس داشته‌باشد.
    علاوه براین، وجود انواع مختلف داده‌های جمع‌آوری شده توسط یک اینترنت اشیاء شهری فراگیر، می‌تواند برای افزایش شفافیت استفاده شود، اقدامات تصمیم‌گیران محلی را ترویج دهد، آگاهی شهروندان راجع به وضعیت شهرشان را بیشتر کند، شهروندان را به مشارکت فعال در مدیریت عمومی تشویق کند و باعث خلق سرویس‌های جدیدی علاوه بر سرویس‌های فراهم شده توسط اینترنت اشیاء بشود \cite{cuff2008urban}.
    بنابراین استفاده از اینترنت اشیاء به صورت خاص برای تصمیم‌گیران محلی و منطقه‌ای جذاب است و باعث می‌شود که آن‌ها اولین استفاده کننده‌های این تکنولوژی‌ها باشند و به عنوان کاتالیزور برای انطباق الگوی اینترنت اشیاء در مقیاس‌های بزرگ‌تر عمل کنند.

  \section{خدمات شهر هوشمند}
    در ادامه برخی از خدماتی که امکان ارائه‌ی آن‌ها توسط اینترنت اشیاء شهری فراهم می‌شود را مرور می‌کنیم.
    \subsection{نظارت بر سلامت ساختار ساختمان‌ها}
      نگهداری مناسب ساختمان‌های تاریخی شهر‌ها نیاز به نظارت مداوم وضعیت واقعی ساختمان‌ها و پیدا‌کردن‌ مکان‌هایی که بیشترین تاثیر را از عوامل خارجی دریافت می‌کنند دارد.
      اینترنت اشیاء شهری می‌تواند یک پایگاه داده‌ی توزیع شده از اندازه‌گیری‌های یکپارچگی ساختاری ساختمان‌ها فراهم ‌کند که به وسیله‌ی حسگر‌های مناسبی که در نقاط مختلف ساختمان نصب شده‌اند، اندازه‌گیری شده‌اند.
      این حسگر‌ها می‌توانند، لرزش و تغییر شکل، رطوبت و دما را اندازه‌گیری کنند و شرایط محیطی را به طور کامل مشخص کنند\cite{lynch2006summary}.
      این پایگاه داده باید بتواند هزینه بالای لازم برای آزمایش دوره‌ای مقاومت ساختمان‌ها توسط عوامل انسانی را کاهش دهد و امکان نظارت فعال بر وضعیت ساختمان‌ها را فراهم کند.
      همچنین امکان ترکیب کردن اطلاعات لرزش ساختمان‌ها و اطلاعات مربوط به زمین لرزه‌های کوچک برای بررسی و فهم تاثیر آن‌ها بر ساختمان‌ها فراهم می‌شود.
      با این وجود تحقق عملی این خدمت، نیاز به نصب حسگر‌هایی در ساختمان‌ها و محیط‌های اطراف و ارتباطشان با یک مرکز کنترل دارد که ممکن است نیاز به یک سرمایه‌گذاری اولیه برای ایجاد این زیرساخت‌ها داشته باشد.

    \subsection{مدیریت پسماند}
      هزینه‌ی بالای خدمات مدیریت پسماند و مشکلات نگهداری پسماند در محل‌های دفن زباله باعث می‌شود که مدیریت پسماند یکی از مهم‌ترین مشکلات در بیشتر شهر‌های بزرگ می‌باشد.
      نفوذ بیشتر راه‌حل‌های مبتنی بر فناوری‌های ارتباطات و اطلاعات در این حوزه می‌تواند به طور چشمگیری باعث کاهش هزینه‌ها بشود و بهبود‌هایی در زمینه‌های اقتصادی و زیست محیط داشته باشد.
      برای مثال استفاده از سطل‌های زباله هوشمند که امکان اندازه‌گیری وزن محتویات درون سطل را دارند و امکان بهینه‌سازی فرآیند جمع‌آوری پسماند را فراهم میکنند، می‌توانند هزینه‌ی جمع‌آوری پسماند را کاهش دهند و کیفیت بازیافت پسماند را بیشتر کنند \cite{nuortio2006improved}.
      برای رسیدن به یک سرویس جمع‌آوری پسماند هوشمند، اینترنت اشیاء باید بتواند سطل‌های زباله هوشمند را به یک مرکز کنترل وصل کند.
      در این مرکز کنترل یک نرم‌افزار بهینه‌سازی اجرا می‌شود تا مدیریت بهینه کامیون‌های جمع‌آوری پسماند را مشخص کند.

    \subsection{نظارت بر کیفیت هوا}
      اتحادیه اروپا به صورت رسمی اهدافی را برای کاهش تغییرات اقلیمی در دهه آینده تعیین کرده است.
      از آن‌ها می‌توان به کاهش ۲۰ درصدی در تولید گاز‌های گلخانه‌ای در سال ۲۰۲۰ میلادی نسبت به سال ۱۹۹۰، کاهش ۲۰ درصدی مصرف انرژی با بهبود بهره‌وری انرژی و افزایش ۲۰ درصدی استفاده از انرژی‌های تجدید‌پذیر را نام برد.
      در زمینه کیفیت هوا، اینترنت اشیاء می‌تواند روش‌هایی برای پایش کیفیت هوای مناطق پر ازدحام و پارک‌ها ارائه دهد \cite{al2010mobile}.
      علاوه بر این‌ها امکانات ارتباطی می‌توانند برای برنامه‌های روی دستگاه‌های هوشمند دوندگان امکان ارتباط با زیرساخت‌های لازم را فراهم کنند.
      به کمک این زیرساخت‌ها، افراد می‌توانند سالم‌ترین مسیر برای فعالیت‌های خارج از منزل را پیدا کنند.
      تحقق خدماتی از این قبیل نیازمند نصب حسگر‌های آلودگی و کیفیت هوا در نقاط مختلف شهر و قرارگیری داده‌های این حسگر‌ها به صورت عمومی در اختیار شهروندان است.

    \subsection{نظارت بر آلودگی صوتی}
      مسئولین شهری معمولا قوانینی برای کاهش آلودگی صوتی در مراکز شهر برای برخی ساعات تصویب می‌کنند.
      اینترنت اشیاء شهری می‌تواند یک سرویس نظارت بر آلودگی صوتی ارائه دهد که وظیفه‌ی آن اندازه‌گیری سطح آلودگی صوتی در ساعتی مشخص در مناطقی که این سرویس در آن‌ها برقرار است، باشد \cite{maisonneuve2009citizen}.
      همچنین این سرویس با استفاده از الگوریتم‌های تشخیص صدا می‌تواند شکستن شیشه‌ها و یا صدای نزاع خیابانی را تشخیص دهد و با این کار باعث افزایش امنیت بشود.
      با این حال نصب حسگر‌های صوتی به دلیل مسائل مربوط به حریم شخصی می‌تواند حساسیت بر انگیز باشد.

    \subsection{مدیریت ترافیک}
      از دیگر خدمات شهر هوشمند که به وسیله‌ی اینترنت اشیاء شهری قابل دستیابی است، به نظارت بر ترافیک شهر می‌توان اشاره کرد.
      با این که نظارت ویدیویی ترافیک در حال حاظر در بیشتر شهر‌ها استفاده می‌شود، استفاده از شبکه‌های گسترده کم توان می‌تواند منبع انبوهی از اطلاعات را در اختیار قرار دهد.
      نظارت بر ترافیک می‌تواند با نصب سامانه‌های موقعیت یاب بر روی خودرو‌های جدید تحقق یابد \cite{li2008performance} و با خدمات نظارت بر کیفیت هوا و نظارت بر آلودگی صوتی ترکیب شود.
      این اطلاعات، اهمیت زیادی برای مسئولین و شهروندان خواهد داشت چرا که باعث افزایش نظم و برنامه ریزی بهتر می‌شود.

    \subsection{نظارت بر مصرف انرژی در شهر}
      در کنار نظارت بر کیفیت هوا، اینترنت اشیاء شهری می‌تواند خدماتی برای نظارت بر مصرف انرژی تمام شهر ارائه کند.
      با این کار شهروندان و مسئولین می‌توانند یک گزارش دقیق از میزان انرژی مورد نیاز برای سرویس‌های مختلف (روشنایی عمومی، حمل و نقل، چراغ‌های راهنمایی و رانندگی، دوربین‌های نظارتی، گرمایش و سرمایش ساختمان‌های عمومی و غیره) داشته باشند.
      این کار می‌تواند امکان شناسایی منابع اصلی مصرف انرژی را فراهم سازد تا بتوان برای بهینه‌سازی رفتار آن‌ها چاره‌ای پیدا کرد.
      برای ممکن ساختن این خدمات، دستگاه‌های نظارت بر میزان مصرف انرژی باید با شبکه‌های قدرت یکپارچه بشوند.
      هم چنین امکان کنترل فعال منابع تولید انرژی مانند صفحات خورشیدی می‌تواند باعث بهبود این خدمات بشود.

    \subsection{پارک هوشمند}
      خدمات پارک هوشمند، مبتنی بر حسگر‌های خیابان‌ها و نمایشگر‌های هوشمند است که رانندگان خودرو‌ها را به بهترین جای پارک در شهر هدایت می‌کند \cite{lee2008intelligent}.
      از مزایای این خدمت می‌توان کاهش زمان لازم برای برای پیدا کردن جای پارک و در نتیجه کاهش انتشار گاز CO از ماشین‌ها، کاهش ترافیک و افزایش رضایت شهروندان اشاره کرد.
      این خدمات می‌توانند به صورت مستقیم در زیرساخت اینترنت اشیاء شهری ادغام شوند چرا که تولید کنندگان، دستگاه‌های لازم برای آن را آماده کرده‌اند.
      این خدمات امکان شناسایی افراد معلول و ارائه‌ی خدمت ویژه به آن‌ها را فراهم می‌کند.
      مثلا اجازه استفاده از پارکینگ‌های خاص و یا ارائه‌ی ابزار‌هایی برای مشخص کردن سریع استفاده غیر مجاز از پارکینگ‌ها.

    \subsection{روشنایی هوشمند}
      برای کاهش انرژی مصرفی شهر، استفاده بهینه از روشنایی بسیار مهم است.
      به طور خاص این خدمت می‌تواند شدت روشنایی خیابان‌ها را مطابق ساعت روز، وضعیت هوا و حضور افراد تنظیم کند.
      این خدمت برای این که به درستی کار کند نیاز به این دارد که اطلاعات روشنایی شهر را در زیرساخت شهر هوشمند ترکیب کند.
      از دیگر مزایای آن این است که امکان شناسایی ساده خطا در سامانه روشنایی شهر در این روش وجود خواهد داشت.

    \subsection{اتوماسیون ساختمان‌های عمومی}
      یکی از کاربرد‌های تکنولوژی‌های اینترنت اشیاء نظارت بر مصرف انرژی ساختمان‌های عمومی (مدارس، دفاتر مدیریت عمومی و موزه‌ها) و بهبود شرایط محیطی برای فعالیت افراد است.
      برای این منظور از انواع متفاوتی از حسگر‌ها و فعال‌کننده‌هایی که نور، دما و رطوبت را کنترل می‌کنند استفاده می‌شود.
      با کنترل این پارامتر‌ها، علاوه بر کاهش هزینه‌ی گرمایش و سرمایش، سطح آسایش افرادی که در این محیط‌ها کار می‌کنند افزایش پیدا می‌کند که خود می‌تواند باعث افزایش بهره‌وری افراد شود \cite{lee2008intelligent}.

  \section{سرویس‌ها و وظیفه‌ها}
    سرویس‌های موجود در شهر هوشمند را می‌توان از چند نظر تقسیم بندی کرد. یکی از انواع این تقسیم‌بندی ها تقسیم‌بندی برمبنای کیفیت مورد انتظار(QoS\LTRfootnote{Quality of Service}) است. به این صورت که یک سری معیار به عنوان معیارهای کمی مشخص می‌شود که لازم است این سرویس‌ها به گونه‌ای باشند که این معیارها را تا حد خوبی برآورده سازند. از جمله‌ی این معیارهای کمی می‌توان به تاخیر مربوط به انجام سرویس اشاره کرد. 
    تاخیر هر سرویس خود چند بخش مختلف دارد. در مورد سرویس اطفا حریق می‌توان گفت که ابتدا لازم است که حسگرها اطلاعات پیرامون خود را جمع‌آوری کنند سپس این اطلاعات جمع‌آوری شده به مرکز پردازشی ارسال شود و در آنجا این اطلاعات پردازش شود و نتیجه‌ی اطلاعات به فعال‌کننده ارسال شود و درنهایت آن‌ها شروع به اقدام کنند در تمام این مراحل تاخیرهای مختلف وجود دارد.
    با توجه به رشد خیلی خوب این روزها در ساختار دستگاهِ‌های الکترونیکی می‌توان گفت که دستگاه‌های خیلی دقیق و سریعی در بازار موجود است که با کمترین خطا درحال دریافت اطلاعات و همچنین انجام کارهای مربوطه هستند و بحث مربوط به افزایش بازدهی در این دستگاه‌ها در گرایش‌های مربوط به رشته‌ی الکترونیک می‌تواند بحث جذابی باشد، لذا در این پایان نامه از تاخیرهای مربوط به عملکرد حسگرها و فعال‌کننده‌ها صرف‌‌نظر شده است و فرض براین است که این دستگاه‌ها به محض دریافت اطلاعات بلافاصله کار خود را انجام می‌دهند. اما در مورد انواع دیگر تاخیر لازم است که دقت لازم انجام گیرد که در ادامه به تفصیل بیشتر دراین مورد صحبت خواهد شد.
    
    \subsection{انواع تاخیر در شبکه‌های سوئیچ بسته}
    در این بخش قرار است که انواع مختلف تاخیر در شبکه‌های سوئیچ بسته بررسی شود . 
    در این شبکه‌ها یک بسته ارسالی، سفر خود را از یک میزبان (مبدأ) شروع می‌کند، از تعدادی سوییچ و مسیریاب می‌گذرد و در پایان سفر خود، به میزبان دیگر (مقصد) می‌رسد. با حرکت بسته از یک گره\LTRfootnote{Node} (میزبان یا مسیریاب) به گره دیگر در طول مسیر، در هر گره انواع مختلفی از تاخیر می‌تواند برای بسته در حال ارسال اتفاق بیفتد.
    این تاخیر ها عبارتند از
    \begin{enumerate}
    	\item \textbf{تاخیر پردازش:}
    مدت زمانی است که طول می‌کشد یک گره، بسته دریافتی را تحلیل کند و بفهمد که آیا لازم است آن را به سمت یک گره دیگر هدایت کند و یا اینکه خودش آن را پردازش کند. همچنین اگر زمانی صرف بررسی سالم بودن بیت‌های موجود در بسته شود این زمان جزء این دسته از تاخیر قرار می‌گیرد.
    \item \textbf{تاخیر صف:}
    تاخیری است که بسته‌ها درصف ورودی منتظر می‌شوند تا نوبت پردازش آن‌ها برسد که میزان این تاخیر با توجه به فرمول‌های مربوط به تئوری صف و نوع صف و همچنین مدل پردازشی گره محاسبه می‌شود. همچنین درصورتی که گره موجود، یک گره مسیریاب باشد یعنی عملکرد آن به صورتی باشد که بسته‌های دریافتی را به سایر گره‌ها هدایت می‌کند ممکن است در خروجی این گره نیز یک صف تشکیل شود و بسته در آن صف خروجی هم مقداری منتظر بماند.
    \item \textbf{تاخیر انتقال:} 
    این تاخیر برابر است با مدت زمان لازم برای انتقال کلیه‌ی بیت‌های موجود در بسته بر روی لینک‌های موجود که همانطور که می‌دانیم در حالت ساده شده با فرض اینکه بسته ارسالی شامل $L$ بیت باشد و همچنین آهنگ انتقال لینک به اندازه $R$ بیت بر ثانیه باشد آن‌گاه تاخیر انتقال این بیت به اندازه‌ی$\frac{L}{R}$ ثانیه خواهد بود. 
    \item \textbf{تاخیر انتشار:} 
    به محض آن‌که مسیریاب A یک بیت را روی لینک خروجی فرستاد، این بیت باید تا مسیریاب B منتشر شود. زمان لازم برای منتشر شدن این بیت از ابتدای لینک تا مسیریاب B را تاخیر انتشار می‌گویند.
    سرعت حرکت بیت‌ها روی یک لینک در واقع همان سرعت انتشار امواج الکترومغناطیسی در لینک است، که به نوع رسانه فیزیک مورد استفاده در لینک (فیبر نوری، زوج به هم تابیده، بی سیم و …) بستگی دارد.
    تاخیر انتشار در یک لینک ارتباطی برابر است با فاصله بین دو مسیریاب (طول لینک که با $d$ نشان می‌دهند) تقسیم بر سرعت انتشار $s$ یعنی $\frac{d}{s}$.
    \end{enumerate}
    تاخیر انتقال زمان لازم برای بیرون دادن تمامی بیت های یک بسته توسط مسیریاب است و به صورت تابعی از طول بسته و آهنگ انتقال به لینک خروجی در مسیریاب تعریف می‌شود؛ در نتیجه هیچ ارتباطی با فاصله بین دو مسیریاب (یا همان طول لینک) ندارد.
    از طرفی، تاخیر انتشار زمانی است که طول می‌کشد تا یک بیت از یک مسیریاب به مسیریاب بعدی برسد، این تاخیر تابعی از فاصله بین دو مسیریاب و سرعت انتشار امواج الکترومغناطیسی در رسانه فیزیکی لینک می باشد و هیچ ارتباطی با طول بسته یا آهنگ انتقال لینک ندارد.
    معمولا در مقالات مربوط به شبکه‌های بی‌سیم از تاخیر انتشار در مقابل تاخیر انتقال صرف نظر می‌شود و یا این دو تاخیر را به عنوان یک تاخیر در نظر می‌گیرند. در این پایان‌نامه نیز این دو تاخیر به صورت جدا درنظر گرفته نشده‌اند و عملا از تاخیر انتشار صرف‌نظر شده است. 
    
  \section{پردازش لبه}
    گسترش اینترنت اشیاء و موفقیت سرویس‌های ابری باعث ایجاد الگو‌ی جدیدی در پردازش داده‌ها به نام پردازش لبه \LTRfootnote{Edge Computing} شده است.
    در این الگوی پردازشی سعی بر این است که پردازش داده‌ها در لبه شبکه انجام شود.
    طبق براورد‌های انجام شده در \cite{2018cisco} تعداد دستگاه‌های متصل شده به شبکه ۵۰ میلیارد عدد خواهد بود.
    بعضی از کاربردهای اینترنت اشیاء نیاز به زمان پاسخ کوتاه دارند، بعضی ممکن است دارای داده‌های محرمانه وشخصی باشند و بعضی از این کاربرد‌ها می‌توانند بار سنگینی برای شبکه داشته باشند و پردازش ابری ممکن است روش مناسبی برای این کاربرد‌ها نباشد.

    \subsection{چرا به پردازش لبه نیاز داریم؟}
      در حال حاظر نرخ تولید داده در لبه شبکه در حال افزایش می‌باشد.
      بنابراین پردازش این داده‌ها در لبه شبکه روش کارامدتری خواهد بود.
      در ادامه دلایلی برای لزوم استفاده از پردازش لبه را بر می‌شماریم \cite{shi2016edge}:
      \begin{description}
        \item [فشار از سمت پردازش ابری:]
          قرار دادن همه وظایف پردازش در ابر به عنوان یک روش کارآمد برای پردازش داده‌ها شناخته شده است چرا که قدرت پردازشی ابر قابل مقایسه با قدرت پردازشی اشیاء در لبه شبکه نیست.
          با این حال، در مقایسه با پیشرفت سریع سرعت پردازش داده‌ها، پهنای‌باند شبکه‌ها تقریبا ثابت مانده‌اند.
          افزایش تولید داده در لبه شبکه باعث شده است که انتقال این داده‌ها به گلوگاه پردازش ابری تبدیل شود.
          به عنوان مثال یک اتومبیل خودران را در نظر بگیرید.
          در هر ثانیه ۱ گیگابایت داده توسط آن تولید می‌شود و پردازش بلادرنگ\LTRfootnote{Real-Time} این داده‌ها برای تصمیم‌گیری درست لازم است.
          اگر قرار باشد که همه این داده‌ها برای پردازش به ابر فرستاده شوند، زمان پاسخ بسیار طولانی خواهد بود.
          علاوه بر این، پهنای باند و قابلیت اطمینان شبکه‌های فعلی برای پردازش داده‌‌های تعداد زیادی خودرو در یک منطقه کافی نخواهد بود.
          در این موارد داده‌ها باید در لبه شبکه پردازش شوند.
          این کار زمان پاسخ کوتاه‌تر، پردازش کارآمد‌تر و فشار کم‌تر بر شبکه را به ارمغان می‌آورد.

        \item [کشش از سمت اینترنت اشیاء:]
          انتظار می‌رود که تعداد اشیاء لبه شبکه به میلیارد‌ها دستگاه برسد.
          در نتیجه داده‌های خام تولید شده توسط آن‌ها حجم بسیار بزرگی خواهند داشت و این حجم بزرگ باعث می‌شود که روش‌های مرسوم پردازش ابری مناسب برای پردازش این حجم از داده‌ها نباشند.
          در نتیجه می‌توان درنظر گرفت که بیشتر این داده‌های تولید شده در اینترنت اشیاء هیچ وقت برای پردازش به ابر ارسال نمی‌شوند و در لبه شبکه پردازش می‌شوند.

          \cref{fig:chapter_2:cloud_paradigm} ساختار مرسوم در پردازش ابری را نشان می‌دهد.
          تولید کننده‌های داده، داده‌های خام را به ابر انتقال می‌دهند.
          مصرف کننده‌ها هم با ارسال درخواست نتیجه را از ابر به‌دست می‌آورند.
          با این حال این ساختار برای استفاده در اینترنت اشیاء مناسب نیست.
          اولا حجم داده‌ی تولید شده در لبه شبکه بسیار زیاد است که پهنای باند شبکه و منابع پردازشی زیادی را طلب می‌کند.
          ثانیا حفظ حریم شخصی به عنوان یک مانع برای پردازش ابری ایجاد می‌کند.
          در انتها، بسیاری از گره‌های انتهایی در اینترنت اشیاء دارای انرژی محدودی هستند و ماژول‌های مخابرات بی‌سیم معمولا مصرف انرژی بالایی نیاز دارند.
          در نتیجه انجام وظایف پردازشی در لبه شبکه می‌تواند از لحاظ مصرف انرژی بهینه‌ باشد.

          \begin{figure}[h]
            \centerline{\includegraphics[width=10cm]{graphics/chapter_2/cloud_paradigm}}
            \caption{الگوی پردازش ابری \cite{shi2016edge}}
            \label{fig:chapter_2:cloud_paradigm}
          \end{figure}

        \item [تغییر از مصرف کننده داده به تولید کننده داده:]
          در الگوی پردازش ابری، دستگاه‌های انتهایی در لبه شبکه معمولا نقش مصرف کننده داده را دارند.
          به عنوان نمونه می‌توان تماشای یک ویدیو را مثال زد.
          با این حال امروزه مردم در حال تولید داده توسط دستگاه‌هایشان هستند.
          برای مثال امروزه بسیار عادی است که افراد عکس‌ها و ویدیو‌هایی که توسط خودشان ضبط شده است را در سرویس‌های اینترنتی مختلف به اشتراک بگذارند.
          در \cite{2019domo} اشاره شده است که در هر دقیقه در \lr{twitter} ۵۱۱۲۰۰ توییت جدید ارسال می‌شود و در \lr{instagram} ۵۵۱۴۰ عکس جدید بارگذاری می‌شود.
          باید توجه داشت که این تصاویر و ویدیوها می‌توانند حجم زیادی داشته باشند و پهنای باند زیادی را برای بارگذاری استفاده کنند.
          در این موارد ویدیو‌ها باید ابتدا حجمشان در لبه شبکه کاهش پیدا کند تا به وضوح تصویر\LTRfootnote{Resolution} مناسب برای بارگذاری در شبکه برسند.
          به عنوان نمونه‌ی دیگر می‌توان دستگاه‌های مربوط به سلامت را مثال زد.
          داده‌های جمع‌آوری شده توسط این دستگاه‌ها معمولا خصوصی است و پردازش این داده‌ها در لبه شبکه به جای ارسال داده‌ها برای پردازش به ابر به حفظ حریم خصوصی افراد کمک می‌کند.

      \end{description}
    
    \subsection{پردازش لبه چیست؟}
      پردازش لبه به همه تکنولوژی‌هایی اطلاق می‌شود که امکان انجام پردازش‌ داده‌ها در لبه شبکه را فراهم می‌کنند.
      در اینجا منظور از لبه، همه منابع پردازشی و شبکه ای است که بین منبع داده‌ها (جایی که داده‌ها در آن جا تولید می‌شوند) و مراکز داده‌ی ابری قرار دارند.
      برای مثال یک دروازه‌ی شبکه در یک خانه هوشمند، می‌تواند یک منبع پردازشی لبه بین اشیاء و مرکز داده‌ی ابری باشد یا یک مرکز داده‌ی کوچک، می‌تواند یک منبع پردازشی لبه بین دستگاه‌های سیار و ابر در نظر گرفته شود.
      منطق در پردازش لبه این است که پردازش داده‌ها باید در همسایگی منبع داده‌ها انجام شود.
      با این منطق پردازش لبه و پردازش مه دو مفهوم یکسان را خواهند داشت با این تفاوت که پردازش لبه تمرکزش سمت اشیاء است ولی پردازش مه تمرکزش سمت زیرساخت است.

      \begin{figure}[h]
        \centerline{\includegraphics[width=10cm]{graphics/chapter_2/edge_paradigm}}
        \caption{الگوی پردازش لبه \cite{shi2016edge}}
        \label{fig:chapter_2:edge_paradigm}
      \end{figure}
      
      \cref{fig:chapter_2:edge_paradigm} جریان پردازش دو طرفه را در پردازش مه نشان می‌دهد.
      در الگو پردازش لبه، اشیاء نه تنها مصرف کننده داده بلکه تولید کننده‌ داده نیز هستند.
      در واقع در لبه، اشیاء نه تنها می‌توانند سرویس و محتوا از ابر درخواست نمایند بلکه می‌توانند وظایف پردازشی را هم انجام دهند.

    \subsection{مزایای پردازش لبه}
      هدف از پردازش لبه قراردادن پردازش در مجاورت منبع تولید داده‌ها است.
      این کار مزایایی نسبت به روش‌های مبتنی بر پردازش ابری مرسوم دارد.
      در \cite{yi2015fog} برنامه‌ای برای تشخیص چهره ساخته شده است که با انتقال پردازش از ابر به لبه شبکه، زمان پاسخ برنامه از ۹۰۰ به ۱۶۰ میلی ثانیه کاهش پیدا کرده است.
      در \cite{ha2014towards} نویسندگان از واحدهای پردازشی ابری کوچک\LTRfootnote{Cloudlet} برای پردازش وظایف دستیار شناختی پوشیدنی\LTRfootnote{Wearable Cognetive Assitance} استفاده کرده اند و نتایج نشان می‌دهد که زمان پاسخ بین ۸۰ تا ۲۰۰ میلی ثانیه کاهش پیدا کرده است.
      روش ارائه شده در \cite{chun2011clonecloud} برای اجرای برنامه‌های تلفن همراه به کمک استفاده همزمان از پردازش ابری و پردازش لبه توانسته زمان اجرا و انرژی مصرفی را تا ۲۰ برابر کاهش دهد.

    \subsection{کاربرد پردازش لبه در شهر هوشمند}
      ویژگی‌های زیر باعث می‌شوند که پردازش لبه برای استفاده در شهر هوشمند مناسب باشد.
      \begin{enumerate}
        \item \textbf{حجم زیاد داده:}
          تخمین زده می‌شود که در سال ۲۰۱۹ میلادی یک شهر با جمعیت ۱ میلیون نفر، ۱۸۰ پتابایت داده در هر روز تولید می‌کند\cite{index2015forecast}.
          این داده‌ها توسط امنیت عمومی، سلامت، تجهیزات شهری و حمل و نقل تولید می‌شوند.
          ساختن مراکز داده‌ی متمرکز برای رسیدگی به همه‌ی این داده‌ها یک راهکار غیر واقعی است چرا که از این حجم از داده نیاز به قدرت پردازشی بسیار زیادی دارد.
          در این مورد پردازش لبه می‌تواند یک راه حل بهینه برای پردازش این حجم عظیم از داده‌ها باشد.

        \item \textbf{تاخیر کم:}
          برای کاربرد‌هایی که نیاز به تاخیر قابل پیشبینی و کم دارند مانند اورژانس سلامتی و امنیت عمومی، پردازش لبه یک الگوی مناسب است چرا که زمان انتقال داده‌ها را کاهش می‌دهد و ساختار شبکه را ساده‌تر می‌کند
        \item \textbf{آگاهی از مکان:}
          برای کاربرد‌های مبتنی بر موقعیت جغرافیایی مانند حمل و نقل و مدیریت تجهیزات شهری پردازش لبه به دلیل آگاهی از مکان، بهتر از پردازش ابری عمل می‌کند.
      \end{enumerate}
  
  \section{پردازش مه}
  شرکت Cisco برای اولین بار، پردازش مه را در ماه ژانویه سال 2014 معرفی کرد. ایده‌ی پشت پردازش مه نسبت به پردازش ابری، افزایش سرعت پردازش اطلاعات بود. ریشه پردازش مه، همان پردازش ابری است، با این تفاوت که به دلیل سرعت بالاتر انتقال اطلاعات، در فناوری‌هایی مثل اینترنت اشیا مورد استفاده قرار می‌گیرد.
  
  هم‌زمان با رشد و نفوذ دستگاه‌های اینترنت اشیا در حوزه‌های مختلف، استفاده از خدمات پردازش مه نیز در این حوزه اهمیت بیشتری یافت. پیش‌بینی می‌شود فناوری پردازش مه که محاسبات مه هم گفته می‌شود، در آینده رشد قابل توجهی داشته باشد. گزارشی از 451 تحقیق انجام شده در این زمینه نشان داد که بازار پردازش مه تا سال 2022 به ارزشی برابر 4/6 میلیارد دلار خواهد رسید.
  
  \subsection{عملکرد پردازش مه چگونه است؟}
  دستگاه‌هایی که در مه وجود دارند تحت عنوان گره شناخته می‌شوند. هر دستگاه با ارتباط شبکه‌ای، محاسباتی و ذخیره‌سازی می‌تواند یک گره باشد که در هر جایی با یک ارتباط شبکه‌ای می‌توانند قرار گیرند. دستگاه‌های مختلف از کنترل‌کننده‌ها تا مسیریاب‌ها و دوربین‌های فیلمبرداری، می‌توانند به عنوان یک گره مه عمل کنند. این گره‌ها می‌توانند در مناطق هدف مانند دفتر کار یا در یک وسیله نقلیه به‌کار گرفته شوند. وقتی یک دستگاه اینترنت اشیا داده‌هایی تولید می‌کند، این داده‌ها می‌توانند از طریق یکی از این گره‌ها دریافت شوند و در شبکه، پردازش سپس به مراکز داده ابری منتقل ‌شوند.
  
  تفاوت اصلی میان پردازش ابری و پردازش مه این است که محاسبات مه نزدیکی جغرافیایی بیشتری به کاربر نهایی دارد و توزیع جغرافیایی وسیع‌تری ایجاد می‌کند.
  
  \subsection{مزایای پردازش مه}
  دلایل مختلفی برای به‌کارگیری پردازش مه وجود دارد. این دلایل در نهایت باعث افزایش بهره‌وری سازمانی می‌شوند. در ادامه به برخی از مهم‌ترین مزایای بکارگیری پردازش مه می‌پردازیم.
  \begin{itemize}
  \item \textbf{کاهش زمان تاخیر:}
  یکی از بزرگ‌ترین مزایای پردازش مه، کاهش زمان تاخیر است. دیگر لازم نیست داده‌ها برای پردازش به مراکز داده ابری فرستاده شوند و از بین رفتن این مشکل باعث می‌شود که تحلیل و پردازش داده‌ها بسیار بهتر و موثرتر انجام شود.
  
  \item \textbf{کاهش ملزومات عملکردی:}
  عدم ارسال داده‌ها به مراکز داده پردازش ابری علاوه بر صرفه‌جویی در زمان، می‌تواند میزان پهنای باند لازم برای این کار را هم کاهش دهد و در مقابل، این میزان پهنای باند برای ارتباط با حسگرها و مراکز داده ابری بکار برده شود. این روش در نهایت باعث کاهش ملزومات عملکرد می‌شود.
  
  \item \textbf{توزیع جغرافیایی گسترده:}
  استفاده از پردازش مه با تمرکززدایی از شبکه، امکان توزیع جغرافیایی گسترده‌تری را نسبت به شبکه‌سازی سنتی یا پردازش ابری فراهم می‌آورد. این کار به ارایه سرویس با کیفیت‌تر برای کاربر نهایی منجر می‌شود.
  
  \item \textbf{تحلیل در لحظه:}
  در بسیاری از محیط‌ها، توانایی تحلیل فوری داده‌ها، اهمیت بسیار زیادی دارد. حذف عوامل ناکارآمدی و تاخیر که در سرویس‌های خدمات ابری وجود دارند، به معنای آن است که کاربر می‌تواند تحلیل معتبر و لحظه‌ای داده‌ها را در اختیار داشته باشد.
    \end{itemize}
  
  \section{انواع لایه در شبکه}
  با توجه به مطالب گفته شده می‌توان گفت که لایه‌های شبکه را می‌توان به دو دسته تقسیم کرد. دسته‌ی اول لایه‌های مربوط به تولید و جمع‌آوری اطلاعات و دسته‌ی دوم لایه‌های مربوط به پردازش اطلاعات.
  در این پایان‌نامه لایه‌های مربوط به دسته‌ی اول را یک لایه در نظر می‌گیریم که شامل کلیه‌ی حسگرها و فعال‌کننده‌هاست. در مورد دسته دوم سه لایه‌ی لبه، مه و ابری را در نظر می‌گیریم. که به ترتیب فاصله از لایه‌ی حسگر قرار گرفته‌اند.   
 
  در مورد تفاوت بین لایه‌ لبه و لایه مه علاوه بر بحث تفاوت در فاصله و میزان تاخیر می‌توان گفت که ظرفیت پردازشی در لایه‌ی مه از لایه‌ی لبه بیشتر است. همچنین در این پایان‌نامه یک هزینه پردازشی برای گره‌های محاسباتی در نظر گرفته شده‌است به این صورت که به صورت میانگین هزینه‌ی پردازشی در گره‌های لبه از دولایه دیگر بیشتر است و گره‌های موجود در لایه‌ی ابری ارزان‌ترین هزینه پردازشی را دارند. 
  
  دربعضی از تعاریف لایه‌ی لبه را یک لایه مجازی بر روی لایه‌ی حسگر در نظر می‌گیرند به این صورت که اطلاعات حسگرها فراتر از لایه‌ی آن‌ها نمی‌رود و درهمان لایه پردازش می‌شود مزیتی که این فرض دارد جنبه‌ی امنیت و محافظت از اطلاعات است. اما در این پایان‌نامه لایه‌ی لبه را جدا از لایه‌ی  حسگر درنظر گرفته‌ایم. 
   
  حال صورت مسئله‌ای که ایجاد می‌شود این است که با توجه به حق انتخاب‌های موجود و با توجه به اطلاعات حسگرها که لازم است پردازش شوند مناسب ترین مکان برای پردازش این اطلاعات کجاست.
 
  انتخاب مکان مناسب برای پردازش بستگی به تعریف دقیق صورت مسئله و همچنین تعریف دقیق از کیفیت سرویس‌ها دارد. به عنوان مثال می‌توان هدف را این‌گونه در نظر گرفت که شبکه به صورتی کار کند که اطلاعات حسگرها در کمترین زمان ممکن پردازش شود که در این صورت احتمالا قیدی بر روی هزینه‌ی لازم برای پردازش اطلاعات لازم است که درنظر گرفته شود که در نتیجه‌ی آن، نمی‌توان گفت که شبکه در ارزان‌ترین حالت خود به سر می‌برد.
  در یک حالت دیگر می‌توان صورت مسئله را به‌گونه‌ای طراحی کرد که هزینه‌های کلی شبکه در بهینه‌ترین حالت ممکن باشد. در این صورت لازم است که شرایط مربوط به تاخیر به عنوان قید به صورت مسئله اضافه شوند. در هر دوحالت مدلسازی قبل این بحث به وجود می‌آید که آیا شبکه توان پردازش کلیه‌ی اطلاعات حاصل از لایه‌ی حسگرها را دارد یا نه، که لازم است این قسمت نیز به عنوان قید به مسئله اصلی اضافه شود.  در حالتی دیگر می‌توان صورت مسئله را به این صورت تعریف کرد که تعداد پردازش‌ها بیشترین مقدار ممکن باشد. 

  \section{نو‌آوری‌های پایان نامه}
    دستاورد‌های این پایان‌نامه را می‌توان به صورت زیر خلاصه کرد:
    در بخش اول، صورت مسئله‌ی یافتن مکان مناسب برای پردازش اطلاعات با هدف بهینه کردن هزینه‌های کل شبکه مدل‌سازی شده است و مسئله به صورت یک مسئله بهینه‌سازی غیرخطی ترکیبی عددصحیح نوشته شده‌است.
    در این بخش، از مدل صف برای گره‌های پردازشی استفاده شده‌است که قابلیت عملیاتی را بیشتر می‌کند. در ادامه مسئله خطی‌سازی شده که با این‌کار این امکان وجود دارد که مسئله به کمک حل‌کننده‌های موجود به راحتی قابل استفاده باشد. سپس یک راه‌حل غیرمتمرکز ارائه شده‌است. این راه‌حل این امکان را ارائه می‌دهد که هر گره پردازشی قسمتی از حل مسئله را انجام دهد و به کمک توان پردازشی موجود در گره‌های پردازشی مسئله اصلی حل شود. در این بخش لازم است یک اپراتور به عنوان هماهنگ‌کننده\LTRfootnote{Coordinator} وجود داشته باشد. 

    در بخش دوم، ابتدا یک راه‌حل توزیع‌شده برای مسئله خطی‌شده ارائه می‌شود در این راه‌حل نیازی به وجود هماهنگ کننده نیست و تمام گره‌های پردازشی موجود به صورت هماهنگ مسئله تخصیص منبع را حل می‌کنند. در ادامه یک راه‌حل اکتشافی برای صورت مسئله غیرخطی اولیه نوشته شده‌است که بدون نیاز به خطی‌سازی مسئله جواب سریعی ارائه می‌دهد. 


  \section{ساختار پایان‌نامه}
    ساختار پایان‌نامه به شرح زیر است.

    \cref{chap:literature_review} به مروری بر مطالعات انجام شده در زمینه تخصیص منابع اینترنت اشیاء می‌پردازد.
    
    \cref{chap:3-system_model_centralized_decentralized} به بررسی مسئله اختصاص منابع پردازشی باتوجه به مدل ارائه شده می‌پردازد.
    در این فصل مسئله بهینه‌سازی همراه با تمام قیدهای موجود نوشته می‌شود، درنهایت مسئله خطی‌سازی می‌شود، به صورتی‌که حل کردن آن راحت‌تر باشد. 
    درادامه مسئله به صورت غیرمتمرکز بازنویسی می‌شود و یک الگوریتم غیرمتمرکز ارائه می‌شود. 

    در \cref{chap:4-heuristic_distributed} مسئله تخصیص منابع پردازشی به دو صورت دیگر مورد بررسی قرار گرفته است.
    ابتدا یک الگوریتم توزیع‌شده ارائه می‌شود و مجددا مسئله توسط آن الگوریتم حل و بررسی می‌شود. 
    در ادامه یک الگوریتم اکتشافی مبتنی بر روش ویتربی\LTRfootnote{Viterbi} ارائه می‌شود، مزیت این روش این است که نیازی نیست خطی سازی انجام شود و از اولین فرم مسئله که غیرخطی است، می‌توان استفاده کرد. 

    در انتها در \cref{chap:conclusion} به بیان نتیجه‌گیری و کار‌های آینده می‌پردازیم.