\chapter{نتیجه‌گیری و کار‌های آینده}\label{chap:conclusion}
  \thispagestyle{empty}
  در این فصل ابتدا مروری داریم بر کار‌های انجام شده در این پایان نامه و سپس برخی از جهت‌های تحقیقاتی ممکن برای ادامه‌ی این فعالیت پژوهشی بیان خواهند شد.

  \section{خلاصه و جمع‌بندی}
    در این پایان‌نامه، ابتدا به معرفی اینترنت اشیاء و کاربرد‌های آن در شهر هوشمند پرداختیم.
    پس از آن برخی خدمات شهر هوشمند را برشمردیم.
    سپس به معرفی پردازش لبه و مه و دلایل برتری آن‌ها نسبت به پردازش ابری پرداختیم.
    در ادامه کار‌های انجام شده در اختصاص منابع در شبکه اینترنت اشیاء را بررسی کردیم.

    در \cref{chap:3-system_model_centralized_decentralized} صورت مسئله تخصیص منابع جهت پردازش وظیفه‌ها به صورت یک مسئله بهینه‌سازی مدل‌سازی شد. 
    در این نوع از تخصیص منابع، منابع پردازشی به تعدادی وظیفه اختصاص پیدا می‌کنند.
    در این فصل مسئله تخصیص منابع به صورت یک مسئله بهینه سازی فرمول بندی شد که هدف آن کمینه کردن مجموع هزینه پردازشی در کل گره‌های شبکه است.
    با توجه به حل دشوار مسئله مورد نظر در ابتدا این مسئله خطی‌سازی شد. جواب بهینه مسئله به روش متمرکز به دست آمد.
    در ادامه یک روش غیرمتمرکز برای حل مسئله ارائه شد و در نهایت میزان پیچیدگی و نحوه همگرایی در این روش بررسی شد. در آخر نتایج مقایسه‌ای برای این دو روش ارائه شد.

    در \cref{chap:4-heuristic_distributed} ابتدا روش توزیع‌شده برای حل مسئله خطی‌‌شده ارائه شد. این روش جواب زیربهینه ارائه می‌کرد.
    در ادامه صورت مسئله غیرخطی اولیه بازنویسی شد.
    به کمک الگوریتم ویتربی یک راه‌حل اکتشافی به نام VTP برای حل مسئله اولیه به صورت زیر‌بهینه ارائه شد.
    پس از بررسی همگرایی و پیچیدگی دو الگوریتم ارائه شده، شبیه‌سازی‌هایی برای بررسی آن‌ها ارائه شد.
    درنهایت چهار روش موجود در این پایان نامه از نظر زمان رسیدن به جواب مقایسه شد که مشخص شد روش VTP سریع‌ترین روش است. 

  \section{کار‌های آینده}
    در این قسمت چند پیشنهاد برای ادامه این کار مطرح می‌کنیم.
  
  اولین پیشنهاد که به صورت یک پیشنهاد عملی است این است که دو روش غیرمتمرکز و توزیع‌شده به عنوان قراردادهای هوشمندی در شبکه زنجیره‌‌بلوک\LTRfootnote{Blockchain} عملیاتی شوند. برای این‌کار لازم است که کلیه گره‌های موجود در شبکه به عنوان یک گره در شبکه زنجیره‌بلوک در نظر گرفته‌شوند. برای ایجاد امنیت بیشتر در این شبکه می‌توان از زنجیره‌های بلوک خصوصی استفاده کرد.
  
    به عنوان پیشنهاد دوم، با ترکیب کردن این مدل‌ها با شبکه‌های مبتنی بر نرم‌افزار\LTRfootnote{Software Defined Networks (SDN)} می‌توان به مدل بهتری رسید.
    همچنین ترکیب این کار با تخصیص منابع رادیوی برای انتقال داده‌های حسگر‌ها و فعال کننده‌ها هم می‌تواند به واقعی‌تر شدن مدل ارائه شده کمک کند.

    به عنوان پیشنهاد آخر، در مدل ارائه شده در \cref{chap:3-system_model_centralized_decentralized} می‌توان مسئله را به صورت یک بازی\LTRfootnote{Game} تعریف کرد و از روش‌های حل مسئله برای این دسته از مسائل استفاده کرد. 