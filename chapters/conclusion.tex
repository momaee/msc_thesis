\chapter{نتیجه‌گیری و کار‌های آینده}\label{chap:conclusion}
  \thispagestyle{empty}
  در این فصل ابتدا مروری داریم بر کار‌های انجام شده در این پایان نامه و سپس برخی از جهت‌های تحقیقاتی ممکن برای ادامه‌ی این فعالیت پژوهشی بیان خواهند شد.

  \section{خلاصه و جمع‌بندی}
    در این پایان‌نامه، ابتدا به معرفی اینترنت اشیاء و کاربرد‌های آن در شهر هوشمند پرداختیم.
    پس از آن برخی خدمات شهر هوشمند را برشمردیم.
    سپس به معرفی پردازش لبه و مه و دلایل برتری آن‌ها نسبت به پردازش ابری پرداختیم.
    در ادامه کار‌های انجام شده در اختصاص منابع در شبکه اینترنت اشیاء را بررسی کردیم.

    در \cref{chap:system_model_centralized_decentralized} صورت مسئله تخصیص منابع جهت پردازش وظیفه‌ها به صورت یک مسئله بهینه‌سازی مدل‌سازی شد. 
    در این نوع از تخصیص منابع، منابع پردازشی به تعدادی وظیفه اختصاص پیدا می‌کنند و هر سرویس هم از یک منبع پردازشی استفاده می‌کند.
    در این فصل مسئله تخصیص منابع به صورت یک مسئله بهینه سازی فرمول بنده شده که هدف آن کمینه کردن مجموع هزینه سرویس‌ها است.
    با توجه به حل دشوار مسئله مورد نظر یک الگوریتم مبتنی بر مزایده معرفی شده است که در زمان منطقی به جوابی قابل قبول برای مسئله می‌رسد.
    پس از بررسی همگرایی و پیچیدگی راه حل ارائه شده، شبیه‌سازی‌هایی برای بررسی الگوریتم ارائه شده آورده شده است.

    در \cref{chap:many_to_many_allocation} تخصیص منابع پردازشی به صورت چند به چند مورد بررسی قرار گرفته است.
    در این نوع از تخصیص منابع، منابع پردازشی می‌توانند به چند سرویس اختصاص پیدا کنند و سرویس‌ها هم برحسب نیاز می‌توانند از چند منبع پردازشی استفاده کنند.
    در این فصل هم مانند \cref{chap:one_to_one_allocation} تخصیص منابع به صورت یک مسئله بهینه سازی فرمول بندی شده که هدف آن کمینه کردن هزینه همه سرویس‌ها است.
    در این جا هم حل بهینه مسئله بسیار دشوار است به همین دلیل یک الگوریتم زیر بهینه که در زمان معقولی به نتیجه می‌رسد معرفی شده است.
    پس از بررسی همگرایی و پیچیدگی الگوریتم، شبیه سازی‌هایی برای بررسی الگوریتم ارائه شده آورده شده است.

  \section{کار‌های آینده}
    در این قسمت چند پیشنهاد برای ادامه این کار مطرح می‌کنیم.
  
    به عنوان پیشنهاد اول می‌توان تاخیر را برای ارائه دهندگان سکو بررسی کرد.
    در این حالت منابع پردازشی توسط ارائه دهندگان سکو به طور کامل استفاده می‌شوند.
    بر خلاف مورد بررسی شده در این پایان نامه، نرخ ورود داده‌های حسگر‌ها، دیگر ثابت نخواهد بود و میانگین آن برابر میانگین نرخ ورود داده‌های حسگر‌های سرویس‌هایی خواهد بود که از منبع پردازشی استفاده می‌کنند.
    در این حالت هر منبع پردازشی مانند یک سرویس دهنده عمل می‌کند و داده‌های سرویس‌های مختلفی که از آن منبع پردازشی استفاده می‌کنند در یک صف مشترک قرار می‌گرند تا نوبت پردازششان فرا برسد.
    هم‌چنین می‌توان حالتی را بررسی کرد که در شبکه چند ارائه دهنده سکو وجود دارند و آن‌ها هم باید بهترین منبع پردازشی را انتخاب کنند تا به سرویس‌ها ارائه دهند.

    به عنوان پیشنهاد دوم، با ترکیب کردن این مدل‌ها با شبکه‌های مبتنی بر نرم‌افزار\LTRfootnote{Software Defined Networks (SDN)} می‌توان به مدل بهتری رسید.
    همچنین ترکیب این کار با تخصیص منابع رادیوی برای انتقال داده‌های حسگر‌ها و فعال کننده‌ها هم می‌تواند به واقعی‌تر شدن مدل ارائه شده کمک کند.

    به عنوان پیشنهاد آخر، به کمک مدل ارائه شده در \cref{chap:many_to_many_allocation} می‌توان نحوه‌ی مناسب تقسیم کردن منابع پردازشی برای مدل ارائه شده در \cref{chap:one_to_one_allocation} را بدست آورد.
    در واقع در محیط‌های آزمایش و کوچک از مدل ارائه شده در \cref{chap:many_to_many_allocation} استفاده می‌کنیم تا چندین ظرفیت ثابت برای منابع پردازشی بدست بیاوریم. سپس در شبکه‌های بزرگ، منابع پردازشی با استفاده از نتایج بدست آمده، ظرفیت پردازشی خود را تقسیم می‌کنند و با استفاده از روش \cref{chap:one_to_one_allocation} که بسیار سریع‌تر است آن‌ها را به اشتراک می‌گذارند.