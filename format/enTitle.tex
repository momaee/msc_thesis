% !TeX root=main.tex
% در این فایل، عنوان پایان‌نامه، مشخصات خود و چکیده پایان‌نامه را به انگلیسی، وارد کنید.

%%%%%%%%%%%%%%%%%%%%%%%%%%%%%%%%%%%%

\begin{latin}
  %\latinfaculty{}
  \latindepartment{School of Electrical and Computer Engineering}
  \latinsubject{Electrical Engineering}
  \latinfield{Communication Engineering}
  \latintitle{Distributed Intelligence in IoT Networks}
  \firstlatinsupervisor{Dr. Vahid Shah-Mansouri}
  %\secondlatinsupervisor{Second Supervisor}
  %\firstlatinadvisor{First Advisor}
  %\secondlatinadvisor{Second Advisor}
  \latinname{Mohammad}
  \latinsurname{Mahmoodian}
  \latinthesisdate{October 2020}
  
  \latinkeywords{Internet of things (IoT), Smart city, Edge computing, Fog computing, Cloud computing, Distributed optimization, Resource allocation}
  
  \en-abstract{
	Smart city is one of the most important applications of IoT\LTRfootnote{Internet of things} which attracts lots of attention from industry and academy. The main purpose of smart city is to provide the possibility of Internet connection for all of the devices in the network. The generated data of these devices should be processed by the network. Therefore, all of the generated data is transmitted to the processing nodes and the result of the processed data is informed to the devices in form of some actions. One of the important challenges of implementing smart city is the determination of the processing nodes for each generated data. The naive solution for this challenge is processing all of the data in a centralized node which is named cloud. However, this solution is not appropriate in the case of the next-generation network in which there are massive amounts of devices with different processing requirements. To this end, edge computing and fog computing are two of the proposed novel patterns. The most characteristic of these two patterns is the decreasing of the response time to the generated data. However, resource allocation has become challenging when there are different layers including, edge, fog, and cloud for processing the data. Therefore, we have introduced a novel model for resource allocation in smart for the scenario in which there are possible options including, edge, fog, and cloud nodes for processing the data. 
	This problem is modeled as a mixed-integer non-linear programming (MINLP), to minimize the processing cost. According to the NP-Hard nature of the initial optimization problem, the main problem is linearized, and solved by centralized, decentralized, and distributed methods. Due to the necessity of linearization for these methods, a heuristic solution using the idea of the Viterbi algorithm is presented for solving the initial non-linear optimization problem which is named Viterbi-based task placement (VTP). 
	According to the conducted simulations, we indicate that the centralized and decentralized methods converge to the optimal solution. The distributed method with few information exchanges between the devices converges to a sub-optimal solution with the desired error. It should be mentioned that the distributed method can be used in the asynchronous networks. The decentralized and distributed methods unlike the centralized method converge in polynomial time, and therefore, can be used in the network with a massive number of devices. Finally, it is worth noting that the introduced VTP algorithm converges in polynomial time and much faster than the centralized, decentralized, and distributed methods.
  }

  \cleartoleftpage
  \latinabstractPage
  \cleartoleftpage
  \latinfirstPage
\end{latin}
