% !TeX root=main.tex
\faculty{پردیس دانشکده‌های فنی}
\department{دانشکده مهندسی برق و کامپیوتر}
\subject{مهندسی برق}
\field{شبکه‌های مخابراتی}

\title{هوشمندی توزیع‌شده در شبکه‌های اینترنت اشیاء}
\firstsupervisor{دکتر وحید شاه‌منصوری}
%\secondsupervisor{استاد راهنمای دوم}
%\firstadvisor{استاد مشاور اول}
%\secondadvisor{استاد مشاور دوم}
\name{محمد}
\surname{محمودیان}
\studentID{810196391}
\thesisdate{شهریور 1399}
%\projectLabel{پایان‌نامه}
%\degree{}

\firstPage

\cleartorightpage
\besmPage

\cleartorightpage
\firstPage

\cleartorightpage
% \newpage

% \thispagestyle{empty}

%\includepdf{graphics/cert}
% \setcounter{page}{1}
% \vspace{-15cm}
% \begin{center}
%   \vspace{-5cm}
  
%   \makebox[\textwidth]{\vspace{-5cm}\includegraphics[width=\paperwidth]{graphics/cert}}
% \end{center}

 %\incgraph[width=\paperwidth]{graphics/cert}

 \davaranPage
\vspace{.5cm}
% در این قسمت اسامی اساتید راهنما، مشاور و داور باید به صورت دستی وارد شوند
\renewcommand{\arraystretch}{1.2}
 \begin{center}
   \begin{tabular}{|c|p{30mm}|c|c|p{25mm}|c|}
     \hline
     ردیف& سمت& نام و نام خانوادگی& مرتبه \newline دانشگاهی&دانشگاه یا مؤسسه& امضـــــــــــــا \\
     \hline
     ۱& استاد راهنما& دکتر وحید شاه‌منصوری& استاد‌یار& دانشگاه تهران                   &\\
     \hline
     ۲&‌استاد مدعو خارجی&دکتر&دانشیار& دانشگاه \newline صنعتی&                    \\
     \hline
     ۳& داور و نماینده \newline تحصیلات تکمیلی دانشکده & دکتر&استادیار                   & دانشگاه تهران& \\
     \hline
   \end{tabular}
 \end{center}

\cleartorightpage
\esalatPage

\cleartorightpage
\thispagestyle{empty}
{\Large تقدیم به:} \\
\begin{flushleft}
  {
    \huge
    پدر، مادر و خواهرم
  }
\end{flushleft}

\cleartorightpage
\begin{acknowledgementpage}
  سپاس خداوندگار حکیم را که با لطف بی‌کران خود، آدمی را زیور عقل آراست.

  در آغاز وظیفه‌ خود می‌دانم از زحمات بی‌دریغ استاد راهنمای خود، جناب آقای دکتر وحید شاه‌منصوری، صمیمانه تشکر و قدردانی کنم که قطعاً بدون راهنمایی‌های ارزنده‌ ایشان، این مجموعه به انجام نمی‌رسید.

  در پایان، بوسه می‌زنم بر دستان خداوندگاران مهر و مهربانی، پدر و مادر عزیزم و بعد از خدا، ستایش می‌کنم وجود مقدس‌شان را و تشکر می‌کنم از خانواده عزیزم به پاس عاطفه سرشار و گرمای امیدبخش وجودشان، که بهترین پشتیبان من بودند.
% با استفاده از دستور زیر، امضای شما، به طور خودکار، درج می‌شود.
\signature
\end{acknowledgementpage}

\keywords{اینترنت اشیاء، شهرهوشمند، پردازش لبه، تخصیص منابع، راه‌حل غیرمتمرکز، راه‌حل توزیع‌شده}
\fa-abstract{
	شهر هوشمند\LTRfootnote{Smart city} یکی از کاربردهای مهم اینترنت اشیاء(IoT\LTRfootnote{Internet of Things})است که اخیرا مورد توجه محققان قرار گرفته‌است. در شهر هوشمند هدف این است که تمام دستگاه‌های موجود امکان اتصال به شبکه اینترنت را داشته باشند. 
	اطلاعات تولید شده توسط این دستگاه‌ها در ابتدا نیاز به پردازش و تحلیل دارد. از این‌رو باید این اطلاعات به منابع پردازشی ارسال و نتیجه پردازش به صورت تصمیم‌هایی به دستگاه‌های مربوطه بازگشت داده شود. 
	یکی از چالش های مهم در پیاده سازی الگوی شهر هوشمند، مکان و نحوه پردازش اطلاعات دریافتی است. ساده‌ترین راه‌حل‌ ارائه شده برای این چالش، پردازش کلیه‌ اطلاعات در یک محل متمرکز به‌نام فضای ابری\LTRfootnote{Cloud} است. با گسترش تعداد دستگاه‌ها، افزایش حجم اطلاعات و نیازهای پردازشی متفاوت سرویس‌ها، این روش متمرکز پردازش اطلاعات،پاسخگوی نیاز تمام دستگاه‌های شهرهوشمند نخواهد بود؛
	به همین منظور پردازش لبه\LTRfootnote{Edge computing} و پردازش مه\LTRfootnote{Fog computing} به عنوان دو الگوی پردازشی جدید مورد توجه محققان قرار گرفته‌اند. مهم ترین مزیت این دو الگو نسبت به پردازش ابری افزایش سرعت در پاسخ‌دهی به اطلاعات دریافتی است؛ اما با اضافه شدن این دو الگو در کنار پردازش ابری مسئله تخصیص منابع برای پردازش اطلاعات به یک چالش مهم تبدیل شده است.
	لذا در این پایان‌نامه مدلسازی و حل مسئله تخصیص منابع با درنظر گرفتن سه لایه لبه، مه و ابری مورد بررسی قرار گرفته‌است. مسئله اصلی درحالت کلی به صورت یک مسئله بهینه سازی غیرخطی ترکیب عددصحیح\LTRfootnote{Mixed-Integer Linear Programming}، با هدف کمینه کردن هزینه‌های پردازشی مدل‌سازی شده است. 
	باتوجه به NP-hard بودن مسئله بهینه‌سازی اولیه، مسئله مورد نظر خطی‌سازی شده و به سه روش متمرکز، غیرمتمرکز و توزیع‌شده مورد بررسی قرار گرفته‌است.
	باتوجه به اینکه سه روش فوق نیاز به خطی‌سازی مسئله دارند، یک راه‌حل زیر‌بهینه به صورت اکتشافی با کمک از الگوریتم ویتربی (VRTP)\LTRfootnote{Viterbi-based Reliable Task Placement} ارائه شده‌است که مسئله غیرخطی اولیه را مستقیما حل می‌کند.
	باتوجه به شبیه‌سازی‌های انجام شده می‌توان گفت که روش‌های متمرکز و غیرمتمرکز به جواب بهینه می‌رسند. روش توزیع‌شده در تعداد محدودی از تبادل اطلاعات بین گره‌ها به جواب زیر‌بهینه با دقت دلخواه $\epsilon$ می‌رسد. این روش در شبکه‌های غیرهمزمان\LTRfootnote{Asynchronous} نیز به جواب می‌رسد. دو روش غیرمتمرکز و توزیع‌شده بر خلاف روش متمرکز در مدت زمان چند جمله‌ای به جواب می‌رسند و در شبکه‌های خیلی بزرگ به‌خوبی می‌توانند مورد استفاده قرار بگیرند. راه‌حل VRTP نیز در زمان چندجمله‌ای و خیلی سریع‌تر از سه روش دیگر به جواب مسئله غیرخطی اولیه می‌رسد.  
}

\cleartorightpage
\abstractPage

%\mojavezPage
\cleartorightpage
