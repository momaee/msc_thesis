% !TeX root=main.tex
\faculty{پردیس دانشکده‌های فنی}
\department{دانشکده مهندسی برق و کامپیوتر}
\subject{مهندسی برق}
\field{شبکه‌های مخابراتی}

\title{هوشمندی توزیع‌شده در شبکه‌های اینترنت اشیاء}
\firstsupervisor{دکتر وحید شاه‌منصوری}
%\secondsupervisor{استاد راهنمای دوم}
%\firstadvisor{استاد مشاور اول}
%\secondadvisor{استاد مشاور دوم}
\name{محمد}
\surname{محمودیان}
\studentID{810196391}
\thesisdate{شهریور 1399}
%\projectLabel{پایان‌نامه}
%\degree{}

\firstPage

\cleartorightpage
\besmPage

\cleartorightpage
\firstPage

\cleartorightpage
% \newpage

% \thispagestyle{empty}

%\includepdf{graphics/cert}
% \setcounter{page}{1}
% \vspace{-15cm}
% \begin{center}
%   \vspace{-5cm}
  
%   \makebox[\textwidth]{\vspace{-5cm}\includegraphics[width=\paperwidth]{graphics/cert}}
% \end{center}

 %\incgraph[width=\paperwidth]{graphics/cert}

 \davaranPage
\vspace{.5cm}
% در این قسمت اسامی اساتید راهنما، مشاور و داور باید به صورت دستی وارد شوند
\renewcommand{\arraystretch}{1.2}
 \begin{center}
   \begin{tabular}{|c|p{30mm}|c|c|p{25mm}|c|}
     \hline
     ردیف& سمت& نام و نام خانوادگی& مرتبه \newline دانشگاهی&دانشگاه یا مؤسسه& امضـــــــــــــا \\
     \hline
     ۱& استاد راهنما& دکتر وحید شاه‌منصوری& استاد‌یار& دانشگاه تهران                   &\\
     \hline
     ۲&‌استاد مدعو خارجی&دکتر&دانشیار& دانشگاه \newline صنعتی&                    \\
     \hline
     ۳& داور و نماینده \newline تحصیلات تکمیلی دانشکده & دکتر&استادیار                   & دانشگاه تهران& \\
     \hline
   \end{tabular}
 \end{center}

\cleartorightpage
\esalatPage

\cleartorightpage
\thispagestyle{empty}
{\Large تقدیم به:} \\
\begin{flushleft}
  {
    \huge
    پدر، مادر و خواهرم
  }
\end{flushleft}

\cleartorightpage
\begin{acknowledgementpage}
  سپاس خداوندگار حکیم را که با لطف بی‌کران خود، آدمی را زیور عقل آراست.

  در آغاز وظیفه‌ خود می‌دانم از زحمات بی‌دریغ استاد راهنمای خود، جناب آقای دکتر وحید شاه‌منصوری، صمیمانه تشکر و قدردانی کنم که قطعاً بدون راهنمایی‌های ارزنده‌ ایشان، این مجموعه به انجام نمی‌رسید.

  در پایان، بوسه می‌زنم بر دستان خداوندگاران مهر و مهربانی، پدر و مادر عزیزم و بعد از خدا، ستایش می‌کنم وجود مقدس‌شان را و تشکر می‌کنم از خانواده عزیزم به پاس عاطفه سرشار و گرمای امیدبخش وجودشان، که بهترین پشتیبان من بودند.
% با استفاده از دستور زیر، امضای شما، به طور خودکار، درج می‌شود.
\signature
\end{acknowledgementpage}

\keywords{اینترنت اشیاء، شهرهوشمند، پردازش لبه، تخصیص منابع، راه‌حل غیرمتمرکز، راه‌حل توزیع‌شده، قرارداد هوشمند}
\fa-abstract{
	شهرهوشمند\LTRfootnote{Smart city} یکی از کاربردهای مهم اینترنت اشیاء است که اخیرا هم در عرصه تحقیقاتی و هم در عرصه صنعت مورد توجه خاص قرار گرفته‌است. در شهر هوشمند هدف این است که تمام دستگاه‌های موجود امکان اتصال به شبکه اینترنت را داشته باشند، به‌طوری‌که تمام تبادل اطلاعات به صورت کاملا خودکار و در بستر این شبکه انجام شود. 
	اطلاعات تولید شده توسط این دستگاه‌ها در قدم اول نیاز به پردازش و تحلیل دارد، لذا لازم است این اطلاعات در بستر اینترنت به منابع پردازشی ارسال شوند و نتیجه پردازش به دستگاه‌های مربوطه بازگشت داده شود. 
	یکی از چالش های مهم در پیاده سازی الگوی شهر هوشمند، مکان مورد نیاز جهت پردازش اطلاعات دریافتی است. از جمله ساده‌ترین راه‌حل‌های ارائه شده برای این چالش، پردازش کلیه‌ اطلاعات دریافتی در یک محل متمرکز به‌نام فضای ابری\LTRfootnote{Cloud} است. هرچند این روش متمرکز پردازش اطلاعات، با گسترش تعداد دستگاه‌ها و افزایش حجم اطلاعات، همچنین با به وجود آمدن سرویس‌های جدید که زمان پردازشی متفاوتی دارند، پاسخگوی نیاز تمام دستگاه‌های شهرهوشمند نخواهد بود.
	به همین منظور پردازش لبه و پردازش مه به عنوان دو الگو پردازشی جدید مورد توجه محققان قرار گرفته اند. مهم ترین مزیت این دو الگوی پردازشی نسبت به پردازش ابری افزایش سرعت در پاسخ‌دهی به اطلاعات دریافتی است. اما با اضافه شدن این دو الگو پردازشی در کنار پردازش ابری مسئله تخصیص منابع برای پردازش اطلاعات به یکی از مهمترین چالش های اپراتور شبکه در استفاده از شهر هوشمند تبدیل شده است که مدلسازی و حل کردن آن نیازمند توان پردازشی زیادی‌است. پس از حل شدن این مسئله مشخص می‌شود که هر یک از گره‌های پردازشی موجود در شبکه، کدام بخش از اطلاعات موجود را پردازش کند. 
	در این پایان‌نامه مسئله تخصیص منابع مورد بررسی قرار می‌گیرد، مسئله اصلی درحالت کلی به صورت یک مسئله بهینه سازی غیرخطی ترکیب عددصحیح، که هدف آن کمینه کردن کل هزینه‌های شبکه است، مدل‌سازی می‌شود که این مسئله در دسته مسائل NP-hard قرار می‌گیرد. 
	مسئله خطی سازی می‌شود و به سه صورت متمرکز، غیرمتمرکز و توزیع‌شده حل می‌شود. راه‌حل‌های غیرمتمرکز و توزیع شده به صورتی هستند که از توان پردازشی گره‌های پردازشی استفاده می‌شود و نیازی نیست که یک واحد خارجی همه مسئولیت حل مسئله را برعهده بگیرد. این دو راه‌حل، همچنین این قابلیت را دارند که در شبکه زنجیره‌ بلوک، تحت عنوان قرارداد هوشمند پیاده‌سازی شوند.
	یک راه‌حل شبه‌بهینه نیز به صورت اکتشافی ارائه می‌شود که مسئله غیرخطی را مستقیما اما به‌صورت متمرکز حل می‌شود. 
}

\cleartorightpage
\abstractPage

\cleartorightpage
